% TYPE OF DOCUMENTS

% You can define the class and there are some optional parameters as font size, number of columns, paper orientation etc
%\documentclass[options]{class}
\documentclass[11pt]{article}    %possible classes: article, report, book, slides

% PACKAGES

\usepackage{amsmath, amsthm, amssymb}
%\usepackage{fullpage}
%\usepackage{algorithm}
\usepackage{graphicx}
\usepackage{caption}
\usepackage{subcaption}
%\usepackage[colorlinks=false, pdfborder={0 0 0}]{hyperref}
%\usepackage{cleveref}

\title{Example 1}
\author{[your name goes here]}
\date{\today} %you can also set the date you want

\begin{document}


% Create a title based on the information you gave previously
\maketitle

%Can be used with the class type report or book
%\chapter{}

\section{Learning some environments}

Let's see how items work.

\begin{itemize}
	\item This environment gives me bullet points
	\item[*] Can change the bullet point
\end{itemize}

\noindent What if we want numbers or some sort of enumeration?

\begin{enumerate}
	\item This environment gives me numbers
\end{enumerate}


\subsection{Tables and figures}

Let's start ``easy'' with a table.

\begin{table}[h!]
  \begin{center}
    \begin{tabular}{|r |r |r |r |r |r |r |} %specify how many columns, the position of the text in each column, lines between columns
    \hline %horizontal line
    Sun & Mon & Tue & Wed & Thu & Fri & Sat \\
    \hline
    rest & &  & &  &  & rest \\
    \hline
    \end{tabular}
  \end{center}
  \caption{Weekdays}
  \label{table}
\end{table}

What about figures? It is a similar structure. See Figure~\ref{umbrella}.

\begin{figure}[h] %where the figure will stay. options: h=here, t=top, b=bottom
%\centering
	\includegraphics[width=\linewidth]{rainy} %can add option as width
	\caption{I think we need one of those today.}
	\label{umbrella}
\end{figure}

What if we want to have two figures bundled in the same float? We can! See Figure~\ref{two umbrellas}

\begin{figure}
    \centering
    \begin{subfigure}[b]{0.3\textwidth}
        \includegraphics[width=\textwidth]{rainy}
        \caption{Black umbrella}
        \label{black}
    \end{subfigure}
    ~ %add desired spacing between images, e.g. ~, \quad, \qquad, \hfill,  or a blank line to force the subfigure onto a new line
    \begin{subfigure}[b]{0.3\textwidth}
        \includegraphics[width=\textwidth]{rainypurple}
        \caption{Purple umbrella}
        \label{purple}
    \end{subfigure}
     \caption{Maybe we need more than one umbrella.}
     \label{two umbrellas}
\end{figure}


\section{Simple bibliography}

We can use the ``bibliography'' environment. Each item has a key associated to it so we can refer to those in the text. We use the command
cite to so, as for example \cite{les85}. And we can also put multiple keys at once as in \cite{don89,rondon89}.

\begin{thebibliography}{9}

\bibitem{les85} Leslie Lamport, 1985. 
\emph{\LaTeX---A Document Preparation System---User?s Guide and Reference Manual},
Addision-Wesley, Reading.

\bibitem{don89}Donald E. Knuth, 1989. \emph{Typesetting Concrete
Mathematics}, TUGBoat, 10(1):31-36.

\bibitem{rondon89}Ronald L. Graham, Donald E. Knuth, and Ore
Patashnik, 1989. \emph{Concrete Mathematics: A Foundation for
Computer Science}, Addison-Wesley, Reading.
\end{thebibliography}

\end{document}
