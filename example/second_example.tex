% TYPE OF DOCUMENTS

% You can define the class and there are some optional parameters as font size, number of columns, paper orientation etc
%\documentclass[options]{class}
\documentclass[11pt]{article}    %possible classes: article, report, book, slides

% PACKAGES

\usepackage{amsmath, amsthm, amssymb}
%\usepackage{fullpage}
%\usepackage{algorithm}
\usepackage{graphicx}
%\usepackage[colorlinks=false, pdfborder={0 0 0}]{hyperref}
%\usepackage{cleveref}

\newcommand{\cibele}[1]{\textbf{{[\small From Cibele: #1]}}}

\title{Example 2}
\author{[your name goes here]}
\date{\today} %you can also set the date you want

\begin{document}

% Create a title based on the information you gave previously
\maketitle

\section{Math Mode}

We must include the \emph{math} packages. Just go to the preamble and uncomment the line with packages amsmath, asthma, amssymb.

The \$ \$ is the simplest way to add math inline. Simple example, $x + 2 = y$. In order to display a equation, we use \$\$ \$\$ as follows
$$ x^2 + 2x + 3 =0.$$

Some example using parenthesis and fraction.

$$\left(\frac{x^2}{y^3}\right)$$

What if I am writing my homework and I need to show a sequence of derivations? Go with the align environment. 

\begin{align}
5x + 7 &= 2 \\
 5x &= 2 -7 \\
 x&= \frac{-5}{5} \nonumber \\
 x &= -1
\end{align}

And if my homework involves matrices? 

\begin{align*}
\begin{matrix} %options are matrix, bmatrix, matrix, and more. What is the difference?
1 & 2 & 3 \\
4 & 5 & 6 \\
7 & 8 & 9
\end{matrix} 
\end{align*}

\subsection{Define a command}

We can define our own commands in Latex. I already defined mine:

\cibele{How does it look?}



\end{document}
